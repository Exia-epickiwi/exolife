\section{Filtres}
	
	\paragraph{Normalisation}
	\label{Normalisation}
	La normalisation est une méthode qui s'applique sur l'histogramme de l'image, afin d'effectuer une transformation affine du niveau de gris des pixels afin que l'image utilise toute la dynamique de représentation, dans notre cas avec des images 8 bits, la dynamique est étiré de 0 à 256 valeurs.


	\paragraph{Egalisation}

	\label{Egalisation}
	Cette méthode consiste à appliquer une transformation sur chaque pixel de l'image. Elle consiste en un ajustement du contraste de l'image qui utilise l'histogramme cumulé afin de chercher à obtenir un histogramme plat. Il a donc fallut construire un histogramme de l'image, puis normalisé cet histogramme. A partir de cet histogramme normalisé on peut construire un histogramme cumulé. A partir de cette histogramme cumulé, on peut enfin calculer chaque pixels de la nouvelle image.

	
	\paragraph{Seuillage}
	\label{Seuillage}

	Le seuillage est une méthode de segmentation d'image, à partir d'une image en niveau de gris, le seuillage d'image va créer une image binaire, comportant uniquement deux valeurs noir ou blanc. Par exemple si un seuil est défini à 120 alors la méthode va remplacer chaque pixel qui a une valeur supérieur à 120 par un pixel qui prendra la valeur 255  (blanc) et si la valeur est inférieur à 120 aors il prendra la valeur 0 (noir).

	\paragraph{Médian}
	\label{Médian}

	C'est un filtre utilisé pour réduire le bruit d'une image. Le principe du filtre est de remplacer chaque pixel par la valeur médiane de son voisinage. Il va considérer les valeurs du voisinage par valeurs croissantes et prendre la valeur médiane, ce qui permet de remplacer des valeurs dites aberrantes par une valeur de consensus entre les valeurs voisines. Ce filtre à l'avantage de respecter les contours et le contraste de l'image.

	\paragraph{Convolution}
	\label{Convolution}

	La convolution est l'opérateur de base du traitement linéaire des images. Pour calculer une convolution, on remplace la valeur de chaque pixel par la valeur du produit scalaire entre les valeurs du voisinage du pixel considéré.

	\paragraph{Fourier}
	\label{Fourier}

	Une transformée de Fourier rapide a été utilisée, avec l'algorithme de Cooley-Tukey qui divise l'image afin d'appliquer récursivement la transformation de Fourier sur chaque ligne de la matrice représentant l'image.

	\paragraph{Contour}
	
	Un contour est une variation brusque d'intensité, par exemple, lorsque une zone de l'image passe du blanc au noir brusquement. Nous avons utilisé le filtre de Sobel pour détecter les contours.

	\paragraph{Filtre de Sobel}

	Ce filtre est un opérateur utilisé pour la détection de contours, il calcule le gradient de l'intensité de chaque pixel ce qui permet de trouver les pixels de changement soudain de luminosité. Cet opérateur utilise des matrices de convolution de taille 3x3.

Cette première matrice correspond au filtre horizontal.
$
	\begin{bmatrix}
		-1 & 0 & 1 \\
		-2 & 0 & 2 \\
		-1 & 0 & 1
	\end{bmatrix}
$

Cette deuxième matrice correspond au filtre vertical.
$
	\begin{bmatrix}
		-1 & -2 & -1 \\
		 0 & 0 & 0 \\
		 1 & 2 & 1
	\end{bmatrix}
$

Ensuite les gradients horizontaux et verticaux peuvent être combinés pour obtenir une approximation de la norme du gradient grâce à la formule suivante : 

$G=\sqrt{Gx^2+Gy^2}$
\section{Filtres}
	
	\paragraph{Normalisation}

	La normalisation est une méthode qui s'applique sur l'histogramme de l'image, afin d'effectuer une transformation affine du niveau de gris des pixels afin que l'image utilise toute la dynamique de représentation, dans notre cas avec des images 8 bits, la dynamique est étiré de 0 à 256 valeurs.


	\paragraph{Egalisation}

	Cette méthode consiste à appliquer une transformation sur chaque pixel de l'image. Elle consiste en un ajustement du contraste de l'image qui utilise l'histogramme cumulé afin de chercher à obtenir un histogramme plat. Il a donc fallut construire un histogramme de l'image, puis normalisé cet histogramme. A partir de cet histogramme normalisé on peut construire un histogramme cumulé. A partir de cette histogramme cumulé, on peut enfin calculer chaque pixels de la nouvelle image.
	
	\paragraph{Seuillage}
	
	\paragraph{Median}
	
	\paragraph{Convolution}
	La convolution est l'opérateur de base du traitement linéaire des images. Pour calculer une convolution, on remplace la valeur de chaque pixel par la valeur du produit scalaire entre les valeurs du voisinage du pixel considéré.
	
	\paragraph{Fourier}
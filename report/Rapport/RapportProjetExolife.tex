\documentclass[12pt]{article}

\usepackage[utf8]{inputenc}
\usepackage[T1]{fontenc}
\usepackage[francais]{babel}
\usepackage{setspace}
\usepackage{graphicx}
\usepackage{ulem}
\usepackage{url}
\usepackage{multicol}
\usepackage[top=1cm,left=1cm,right=1cm,bottom=2cm]{geometry}
\usepackage{amsmath}
\usepackage{vwcol}
\usepackage[hidelinks]{hyperref}


\begin{document}

\begin{figure}
	\centering
	\includegraphics[scale=0.8]{images/logo.jpg}
\end{figure}

\hspace{\fill}

\begin{center}
	\LARGE{\textbf{Projet ExoLife}}
\end{center}


\begin{center}
\textbf{Analyse d'images pour but de trouver une source de vie dans l'univers, autre que sur Terre.}\\
\end{center}

\vspace{\fill}

\begin{flushright}
	\bsc{Chiaverini} Marie\\
	\bsc{Blochet} Tanguy\\
	\bsc{Saclier} Baptiste\\
\end{flushright}

\clearpage

\tableofcontents

\section{Introduction}

Les missions pour la recherche d’une vie dans l’univers, ailleurs que sur Terre, existent depuis un certain nombre d’années, mais aucune vie extra-terrestre n’a encore été réellement prouvée. C’est dans ce contexte que se place le projet \emph{ExoLife}, dont la principale mission est l’analyse d’images. Dans le cadre de nos missions, cela s’appliquera en particulier à l’analyse du sol des astres se trouvant à l’intérieur et en dehors de notre système solaire, pour estimer la possibilité d’une vie potentielle.

Pour ce faire, nous avons analysé un certain nombre d'images en nuances de gris pour en déduire un certain nombre d'informations. Les missions que l'on nous a confié ont chacune un objectif différent et nous avons donc effectué différents traitements pour en faire ressortir les informations importantes.

Lors de ce projet, nous avons utilisé le logiciel SciLab permettant de programmer de petits scripts basés sur des matrices de nuaces de gris: les images. Nous avons donc programmé les filtres de base de transformations puis nous avons utilisé ces filtres dans les scripts de mission pour obtenir les résulats finaux.

\clearpage
\subsection{Sous-Mission A1}
\clearpage
	\subsection{Sous-mission A2}
	\begin{vwcol}[widths={0.65,0.2}, rule=0pt]
		\begin{minipage}{0.7\textwidth}
			\paragraph{Objectifs de la mission}

			Déterminer la quantité de méthane dans l'atmosphère de la planète Mars, afin de déterminer si il y a une présence de vie sur cette planète. Pour se faire nous avions à disposition une photographie satellite de la surface de Mars.
		\end{minipage}

		\begin{minipage}{0.3\textwidth}
			\begin{flushright}
				\paragraph{Technique utilisée}
			
				Proportion \&
				Moyenne
			\end{flushright}
		\end{minipage}
	\end{vwcol} 

	\begin{figure}[h]
		\centering
		\includegraphics[scale=0.6]{images/Mars_surface.png}
	\end{figure}
	\vspace{-0.3cm}

	\paragraph{Procédé}	
		Pour remplir la mission nous avons calculé le taux de pixel, puis fais la somme des taux des pixels. Enfin, nous avons fait la moyenne du taux de gaz afin de déterminer la quantité de gaz dans l'atmosphère martienne. Nous n'avons pas eu besoin d'utiliser des filtres.
\clearpage
\subsection{Sous-mission A3}
\clearpage
	\subsection{Sous-mission A4}

	\begin{vwcol}[widths={0.65,0.2}, rule=0pt]
	\begin{minipage}{0.7\textwidth}
	\paragraph{Objectifs de la mission}

	Rendre une image de la planète Jupiter plus nette. Pour se faire nous avions à disposition deux photographies faites à quelques secondes d'intervalle comprenant toutes deux du bruit.
	\end{minipage}

	\begin{minipage}{0.3\textwidth}
	\begin{flushright}
	\paragraph{Filtres utilisés}

	Soustraction \& Filtre médiant
	\end{flushright}
	\end{minipage}

	\end{vwcol} 

	\begin{figure}[h]
	\centering
		\begin{multicols}{2}
		\includegraphics[scale=0.325]{images/Jupiter.png}
		Avant
		\includegraphics[scale=0.325]{images/JupiterAp.png}
		Après
		\end{multicols}
	\end{figure}
	\vspace{-0.9cm}

	\paragraph{Procédé}	
		Le résultat ci-dessus, nous avons utilisé deux filtes à la suite. Tout d'abord, nous avons \emph{soustrait} les deux images ensembles pour obtenir une troisième image ne comprenant que le bruit. Nous avons alors \emph{soustrait} ce résultat a l'image d'origine. Nous obtenons ensuite une image de meilleure qualité mais tout de même bruitée. Pour résoudre ce problème nous avons utilisé un \emph{filtre médiant} permettant de retirer le bruit sans altérer la netteté de l'image.
\clearpage
\section{Mission B}
	\subsection{Sous-mission B1}

	\begin{vwcol}[widths={0.8,0.2}, rule=0pt]
	\begin{minipage}{0.7\textwidth}
	\paragraph{Objectifs de la mission}

	L'objectif de cette mission était de travailler l'image de Gliese 667Cc afin de faire apparaître son atmosphère sur les images prises par une sonde, cette image étant de mauvaise qualité. 
	\end{minipage}
	\begin{minipage}{0.2\textwidth}
		\begin{flushright}
			\paragraph{Filtre utilisé}
		Egalisation
		\end{flushright}
	\end{minipage}
	\end{vwcol} 

	\begin{figure}[h]
	\centering
		\begin{multicols}{2}
		\includegraphics[scale=0.525]{images/Gliese667Cc.png}
		Avant
		\includegraphics[scale=0.525]{images/Gliese667CcAFTER.png}
		Après
		\end{multicols}
	\end{figure}
	\vspace{-0.9cm}

		\paragraph{Procédé}	
			Pour cette mission une \emph{égalisation} a été réalisée sur cette image. Cette méthode fait nettement apparaître l'atmosphère de la planète. Une \emph{normalisation} avait été réalisée en premier lieu, qui faisait aussi apparaître cette atmosphère mais moins nettement.
\clearpage
\subsection{Sous-mission B2} 

	\begin{vwcol}[widths={0.8,0.2}, rule=0pt]
	\begin{minipage}{0.7\textwidth}
	\paragraph{Objectifs de la mission}

	L'objectif de cette mission était d'améliorer la visibilité de l'image afin de la donner à un autre service pour identifier la position d'une naine blanche située à 150 années lumière de la Terre.
	\end{minipage}
	\begin{minipage}{0.2\textwidth}
		\begin{flushright}
			\paragraph{Filtre utilisé}

			Normalisation\up{\ref{Normalisation}}
		\end{flushright}
	\end{minipage}
	\end{vwcol} 

	\begin{figure}[h]
	\centering
		\begin{multicols}{2}
		\includegraphics[scale=0.525]{images/GD61.png}
		Avant
		\includegraphics[scale=0.525]{images/GD61AFTER.png}
		Après
		\end{multicols}
	\end{figure}
	\vspace{-0.9cm}

		\paragraph{Procédé}	
			Pour cette mission une normalisation a été réalisée, faisant clairement apparaître les différentes planètes et étoiles.
\clearpage
\subsection{Sous-mission B3}

	\begin{vwcol}[widths={0.65,0.2}, rule=0pt]
	\begin{minipage}{0.7\textwidth}
	\paragraph{Objectifs de la mission}

	Mettre en valeur les 4 étapes de températures à la surface de la planète. 
	\end{minipage}

	\begin{minipage}{0.25\textwidth}
	\begin{flushright}
	\paragraph{Techniques utilisées}
	
	Seuillage avancé \& Addition
	\end{flushright}
	\end{minipage}

	\end{vwcol} 

	\begin{figure}[h]
	\centering
		\begin{multicols}{2}
		\includegraphics[scale=0.55]{images/HD215497.png}
		Avant

		\includegraphics[scale=0.55]{images/MissionB3.png}
		Après
		\end{multicols}
	\end{figure}
	\vspace{-0.9cm}

	\paragraph{Procédé}	
		Nous avons choisi de mettre en avant les différentes étapes de thempératures par des plages de couleur spécifiques. Pour se faire nous avons effectué un \emph{seuillage avancé} individuel pour chaque plages de température. Nous avons ensuite effectué une \emph{addition} de chacun des résultats pour obtenir l'image finale. 
\clearpage
\section{Mission X}
\subsection{Sous-mission X1}
\clearpage
\subsection{Sous-mission X2}
\clearpage
\subsection{Sous-mission U1}
\clearpage
\subsection{Sous-mission U2}
\clearpage
\section{Filtres}
	
	\paragraph{Normalisation}
	\label{Normalisation}
	La normalisation est une méthode qui s'applique sur l'histogramme de l'image, afin d'effectuer une transformation affine du niveau de gris des pixels afin que l'image utilise toute la dynamique de représentation, dans notre cas avec des images 8 bits, la dynamique est étiré de 0 à 256 valeurs.


	\paragraph{Egalisation}

	\label{Egalisation}
	Cette méthode consiste à appliquer une transformation sur chaque pixel de l'image. Elle consiste en un ajustement du contraste de l'image qui utilise l'histogramme cumulé afin de chercher à obtenir un histogramme plat. Il a donc fallut construire un histogramme de l'image, puis normalisé cet histogramme. A partir de cet histogramme normalisé on peut construire un histogramme cumulé. A partir de cette histogramme cumulé, on peut enfin calculer chaque pixels de la nouvelle image.

	
	\paragraph{Seuillage}
	\label{Seuillage}

	Le seuillage est une méthode de segmentation d'image, à partir d'une image en niveau de gris, le seuillage d'image va créer une image binaire, comportant uniquement deux valeurs noir ou blanc. Par exemple si un seuil est défini à 120 alors la méthode va remplacer chaque pixel qui a une valeur supérieur à 120 par un pixel qui prendra la valeur 255  (blanc) et si la valeur est inférieur à 120 aors il prendra la valeur 0 (noir).

	\paragraph{Médian}
	\label{Médian}

	C'est un filtre utilisé pour réduire le bruit d'une image. Le principe du filtre est de remplacer chaque pixel par la valeur médiane de son voisinage. Il va considérer les valeurs du voisinage par valeurs croissantes et prendre la valeur médiane, ce qui permet de remplacer des valeurs dites aberrantes par une valeur de consensus entre les valeurs voisines. Ce filtre à l'avantage de respecter les contours et le contraste de l'image.

	\paragraph{Convolution}
	\label{Convolution}

	La convolution est l'opérateur de base du traitement linéaire des images. Pour calculer une convolution, on remplace la valeur de chaque pixel par la valeur du produit scalaire entre les valeurs du voisinage du pixel considéré.

	\paragraph{Fourier}
	\label{Fourier}

	Une transformée de Fourier rapide a été utilisée, avec l'algorithme de Cooley-Tukey qui divise l'image afin d'appliquer récursivement la transformation de Fourier.

	\paragraph{Contour}
	
	Un contour est une variation brusque d'intensité, par exemple, lorsque une zone de l'image passe du blanc au noir brusquement.

\clearpage

\section{Algorithmes de traitement}

Par besoin de conscision, nous n'avons pas inclus le code de nos algorithmes dans le raport. En revanche, l'ensemble de nos filtres sont disponibles dans notre dossier \texttt{scripts} sur GitHub (\href{https://github.com/Exia-epickiwi/exolife/tree/master/scripts}{\texttt{goo.gl/yA9lvK}}).

L'ensemble des programmes de mission sont aussi disponibles dans le dossier \texttt{workspace} sur GitHub (\href{https://github.com/Exia-epickiwi/exolife/tree/master/workspace}{\texttt{goo.gl/DBfesI}}).

Pour des informations générales sur le projet et tout les fichiers qui lui sont attachés, vous pouvez consulter le dépot général GitHub (\href{https://github.com/Exia-epickiwi/exolife}{\texttt{goo.gl/KPSNnZ}}), vous y trouverez les mission, les images, les algorithmes de filtrage et le code source du rapport.

\section{Bilans personnels}

\paragraph{\bsc{Saclier} Baptiste} Ce projet fut une très bonne experience dans le domaine des mathématiques et une très bonne introduction à l'imagerie, un domaine très intéressant et en plein essor actuellement. Ce fut un bon retour aux sciences avec un bon sujet.

Durant le projet, l'ensemble du groupe était motivé et une bonne utilisation d'outils comme GitHub et Slack ont permis une très bonne cohésion et une bonne communication entre les membres.

\paragraph{\bsc{Chiaverini} Marie}

Cela fut un projet rapide, travailler sur un autre sujet qui est l'imagerie est intéressant, connaître les différents aspects de filtre et méthode est assez enrichissant. Néanmoins, en fin de semaine, j'ai commencé à avoir une certaine lassitude, c'est un sujet auquel je travaillerai rarement. Le projet est assez bien, l'équipe aussi, ils ont été assez patient et m'ont bien aider lors de certaine mission.

\paragraph{\bsc{Blochet} Tanguy}
	Un projet qui fut très intéressant, c'était une bonne chose de découvrir les bases de l'imagerie et de voir l'ampleur des possibilités de ce domaine. Quelques difficultés au niveau des sciences, mais la plus part des notions sont comprises.

\section{Bilan général}
\vspace{-0.5cm}\hspace{0.5cm}\textit{du chef de projet Chiaverini Marie}
\vspace{0.5cm}

Il y a eu une bonne ambiance et une bonne entraide dans l'équipe. Nous avons pas eu de difficulté en ce qui concerne l'organisation. Les tâches ont été terminés à temps. Le travail effectué avec le groupe est satisfaisante.
\end{document}
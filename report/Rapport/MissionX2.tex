\subsection{Sous-mission X2}

	\begin{vwcol}[widths={0.65,0.2}, rule=0pt]
	\begin{minipage}{0.7\textwidth}
	\paragraph{Objectifs de la mission}

	vérifier que ce qui apparaît sur l'image est bien de la végétation. L'image données est très perturbée par un bruit. Nous devons alors améliorer la visibilité en réduisant ce bruit.
	\end{minipage}

	\begin{minipage}{0.25\textwidth}
	\begin{flushright}
	\paragraph{Technique utilisée}
	
	Filtre median
	\end{flushright}
	\end{minipage}

	\end{vwcol} 

	\begin{figure}[h]
	\centering
		\begin{multicols}{2}
		\includegraphics[scale=0.45]{images/Gliese_581d-V2.png}
		Avant

		\includegraphics[scale=0.45]{images/MissionX2v2.png}
		Après
		\end{multicols}
	\end{figure}
	\vspace{-0.9cm}

	\paragraph{Procédé}
	
		Afin d'accomplir cette mission, il nous fallait nettoyer le bruit de l'image. Il était possible d'utiliser le filtre moyenneur et ou le filtre gaussien, mais le filtre médian surpasse ces deux filtres cités ci-dessus en terme de suppression de bruit et de qualité. Le filtre médian, permet de remplacer la valeur du pixel par la valeur médiane de son voisinage.